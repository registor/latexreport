% 流程图 

\documentclass{article} \usepackage[UTF8]{ctex}
\usepackage[left=3cm,right=3cm,bottom=3cm]{geometry}
\usepackage[dvipsnames]{xcolor}
\usepackage{tikz}
\usepackage{amsmath}
\usetikzlibrary{shapes.geometric, arrows}
\usetikzlibrary{positioning}

\tikzstyle{startstop} = [rectangle, rounded corners=1.3em, minimum width=2cm,
minimum height=1cm,text centered, draw=black, fill=gray!2,line width=1pt]

\tikzstyle{io} = [trapezium, trapezium left angle=70, trapezium right
angle=110, minimum width=3cm, minimum height=1cm, inner sep=0.5cm, text
centered, draw=black, fill=gray!2,line width=1pt]

\tikzstyle{process} = [rectangle, rounded corners, minimum width=4cm,
minimum height=1cm, inner sep=0.5cm, text centered, draw=black,
fill=gray!2,line width=1pt]

\tikzstyle{decision} = [diamond, minimum width=3cm, minimum height=1cm,
text centered, draw=black, fill=gray!2]

\tikzstyle{point} = [rectangle, inner sep=0pt,minimum height=2em]

\tikzstyle{arrow} = [thick,->,>=stealth,line width=1pt,color=black!50]
\tikzstyle{line} = [thick,line width=1pt,color=black!50]

\begin{document}

\begin{center}
\begin{tikzpicture}[node distance=2cm]
\node (start) [startstop] {开始};
\node (test) [io, below of=start,align=center, yshift=-0.3cm] {按给定拉力进行校准张拉
  试,控制$20 \leq \xi \leq 50$};
\node (signal) [process, below of=test,align=center,yshift=-0.8cm] {获
  取并处理加速度信号 \\ 识别固有振动频率
  $\mathbf{f}$与频阶 \\
选择最大和最小频阶$f_{nt}$和$f_{nb}$\\ 标记最大频阶为$nt$,最小频阶为$nb$};
\node (xi1) [process, below of=signal,align=center,yshift=-1.2cm] {由频阶$nt$和$nb$查实用模型系数表
  \\ 确定系数$a_{nt}$、$b_{nt}$和$a_{nb}$、$b_{nb}$};

\node (void) [point, right= 2cm of xi1] {进行3-5次张拉试验};

\node (xi2) [process, below of=xi1,align=center,yshift=-0.5cm] {将上述参
  数代入公式计算频比系数: 
  $K_{tb}=\dfrac{\eta_{nt}a_{nb}}{\eta_{nb}a_{nt}}$ };

\node (xi3) [process, below of=xi2,align=center,yshift=-0.5cm] {将上述参
  数代入公式推测相对刚度:
  $ 
  \xi_{tb} = \dfrac{K_{tb}b_{nt}-b_{nb}}{K_{tb}-1} $ };


\node (EIi) [process, below of=xi3,align=center,yshift=-0.3cm] {
  由相对刚度代入公式得拉索截面抗弯刚度:
  $EI = T\dfrac{l^{2}}{\xi_{tb}^{2}}$ };

\node (EI) [io, below of=EIi,align=center,yshift=-0.3cm] {
  求取各次校准试验拉索截面抗弯刚度均值作为最终识别值。}; 

\node (stop) [startstop,below of=EI] {结束};

 \draw [arrow] (start) -- (test);
 \draw [arrow] (test) -- (signal);
 \draw [arrow] (signal) -- (xi1);
 \draw [arrow] (xi1) -- (xi2);
 \draw [arrow] (xi2) -- (xi3);
 \draw [arrow] (xi3) -- (EIi);
 \draw [arrow] (EIi) -- (EI);
 \draw [arrow] (EI) -- (stop);
 \draw [line] (EIi) -| (void);
 \draw [arrow] (void) |- (test);
\end{tikzpicture}
\end{center}

\end{document}
%%% Local Variables:
%%% mode: latex
%%% TeX-master: t
%%% End:
